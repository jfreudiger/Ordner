\documentclass{article}
\usepackage{german}
\usepackage{inputenc}
\title{CS102 \LaTeX \"Ubung}
\date{\today}
\author{Jessica Freudiger}
\begin{document}
\maketitle
\section{Ich war hier}
Florian Zellweger
\section{Das ist der erst Abschnitt}
Hier könnte auch ein anderer Text stehen
\section{Tabelle}
Unsere wichtigsten Daten finden Sie in Tabelle 1
\begin{center}
\begin{tabular}{c|c|c|c}
  • & Punkte erhalten & Punkte m\"oglich & \% \\ 
  \hline
 Aufgabe 1 & 2 & 4 & 0.5 \\  
 Aufgabe 2 & 3 & 3 & 1 \\ 
 Aufgabe 3 & 3 & 3 & 1 \\  
 \end{tabular} 

Tabelle 1: Diese Tabelle kann auch andere Werte beinhalten.
\end{center}
\section{Formeln}
\subsection{Pythagoras}
Der Satz des Pythagoras errechnet sich wie folgt: $a^2+b^2=c^2$. Daraus können wir die Länge der Hypothenuse wie folgt berechnen: $c=\sqrt{a^2+b^2}$
\subsection{Summen}
Wir können auch die Formel für eine Summe angeben:
\begin{center}
$\sum\limits_{i=1}^{n} i=\frac{n*(n+1)}{2}$ 
\end{center}
\end{document}
